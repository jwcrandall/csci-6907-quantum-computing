\documentclass[12pt]{article}
\usepackage[utf8]{inputenc}

\usepackage{fancyhdr}
\pagestyle{fancy}
\fancyhf{}
\fancyhead[r]{}
\fancyhead[l]{Machine Intelligence}
\fancyfoot[r]{\thepage}
\fancyfoot[l]{\sectionmark}

%hw1
\usepackage{gensymb}
%hw2

%hw5
%\usepackage{unicode-math}

\usepackage{array}
\usepackage{tikz}
\usetikzlibrary{arrows, automata}
\usetikzlibrary{positioning}
\usetikzlibrary{calc}

\usepackage{ifthen}
\usepackage{amsmath}
\usepackage{amsthm}

\newtheorem{example}{Example}
\newtheorem{prop}{Proposition}
\newtheorem{thm}{Theorem}
\newtheorem{cor}{Corollary}

\usepackage{mathtools}
\usepackage{ amssymb }
\usepackage{multicol}
\usepackage{enumitem}

%hw07sol.tex
\usepackage{tikz-qtree}
\usepackage{pdflscape}
\usepackage{rotating}

\usepackage{amssymb}
\usepackage{hyperref}

\usepackage{enumerate}% http://ctan.org/pkg/enumerate




\usepackage{algorithm, algpseudocode, color, gensymb, siunitx, soul, subfiles}
\usepackage[a4paper, total={7.5in, 10in}]{geometry}

\title{Introduction to Quantum Computing\\
\large CSCI 6907}
\author{Joseph Crandall}
\date{Spring 2022}

\begin{document}
\maketitle

\href{https://www2.seas.gwu.edu/~simhaweb/quantum/index.html}{Introduction to Quantum Computing}

\href{https://www2.seas.gwu.edu/~simhaweb/quantum/coursework.html}{Coursework}

%\chapter{Review}

\href{https://www2.seas.gwu.edu/~simhaweb/quantum/modules/review/review.html}{Review}

%\section{Linear Algebra Review}
%\subfile{notes/n01.tex}

%\section{Math Review}
%\subfile{notes/n02.tex}

%\section{Computing Review}
%\subfile{notes/n03.tex}

%\subsection{n04}
%\subfile{n04.tex}

%\subsection{n05}
%\subfile{n05.tex}

\section{Module 1 Quantum Strangeness}
\subfile{modules/m01.tex}

%\section{Module 2 Quantum Linear Algebra, Part I}
%\subfile{modules/m02.tex}

%\section{The Single Qubit}
%\subfile{modules/m03.tex}

%\subsection{Module 4: Quantum linear algebra, part II}

%\subsection{Module 5: Multiple qubits, part I}

%\subsection{Module 6: Multiple qubits, part II}

%\subsection{Module 7: Interlude: Principles of quantum mechanics}

%\subsection{Module 8: The EPR paradox, no-cloning}

%\subsection{Module 9: Gates and circuits}

%\subsection{Module 10: Algorithms, part I}

%\subsection{Module 11: Algorithms, part II}

%\subsection{Module 12: Adiabatic quantum computing}

%\subsection{Module 13: Quantum networks}

%\subsection{Module 14: How qubits are built}

%\section{Term Paper}

\end{document}
