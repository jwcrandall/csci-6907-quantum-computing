\input{preamble.tex}
\usepackage{algorithm, algpseudocode, color, gensymb, siunitx, soul, subfiles, verbatim}
\usepackage[a4paper, total={7.5in, 10in}]{geometry}

\title{Introduction to Quantum Computing\\
\large CSCI 6907}
\author{Joseph Crandall}
\date{Spring 2022}

\begin{document}
\maketitle

\href{https://www2.seas.gwu.edu/~simhaweb/quantum/index.html}{Introduction to Quantum Computing}

\href{https://www2.seas.gwu.edu/~simhaweb/quantum/coursework.html}{Coursework}

%\chapter{Review}

\href{https://www2.seas.gwu.edu/~simhaweb/quantum/modules/review/review.html}{Review}

\begin{comment}

\section{Linear Algebra Review}
\subfile{notes/n01_linear_algebra.tex}

\section{Math Review}
\subfile{notes/n02_math.tex}

\section{Computing Review}
\subfile{notes/n03_computing.tex}

\subsection{Notes 04}
\subfile{notes/n04.tex}

\subsection{Notes n05}
\subfile{notes/n05.tex}

\section{Module 1 Quantum Strangeness}
\subfile{modules/m01.tex}

\section{Module 2 Quantum Linear Algebra, Part I}
\subfile{modules/m02.tex}

\section{Module 3 The Single Qubit}
\subfile{modules/m03.tex}

\section{Module 4: Quantum linear algebra, part II}
\subfile{modules/m04.tex}

\section{Module 5: Multiple qubits, part I}
\subfile{modules/m05.tex}

\section{Module 6: Multiple qubits, part II}
\subfile{modules/m06.tex}

\end{comment}

\section{Module 7: Interlude: Principles of quantum mechanics}
\subfile{modules/m07.tex}

\begin{comment}

\section{Module 8: The EPR paradox, no-cloning}
\subfile{modules/m08.tex}

\section{Module 9: Gates and circuits}
\subfile{modules/m09.tex}

\section{Module 10: Algorithms, part I}
\subfile{modules/m10.tex}

\section{Module 11: Algorithms, part II}
\subfile{modules/m04.tex}

\section{Module 12: Adiabatic quantum computing}
\subfile{modules/m04.tex}

\section{Module 13: Quantum networks}
\subfile{modules/m04.tex}

\section{Module 14: How qubits are built}
\subfile{modules/m04.tex}

\section{Term Paper}
\subfile{paper/tp.tex}

\end{comment}

\end{document}
