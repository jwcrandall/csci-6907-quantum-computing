\documentclass[main.tex]{subfiles}
\begin{document}

\url{https://www2.seas.gwu.edu/~simhaweb/quantum/modules/module2/module2.html}

\begin{enumerate}

\item[] \textbf{In-Class Exercise 1:} Review \href{https://www2.seas.gwu.edu/~simhaweb/quantum/modules/module2/problems2.html#complex}{these examples} and then solve:

    \begin{enumerate}
        \item[a.] Compute $|z|$ when $z=4-3 i$
        \item[b.] Express $3-3 i$ in polar form.
        \item[c.] Show that Im $(2 i(1+4 i)-3 i(2-i))=-4$
        \item[d.] Write $\frac{1}{(1-i)(3+i)}$ in $a+ib$ form.
        \item[e.] If $z=i^{\frac{1}{3}}$, what is the phase of $z^{*}$ in degrees?
    \end{enumerate}

\item[] \textbf{In-Class Exercise 2:} Suppose $z=z^{*}$ for a complex number $z$. What can you infer about the imaginary part of $z$?

\item[] \textbf{In-Class Exercise 3:} Prove the two results stated above, and one more

    \begin{enumerate}
        \item[a.] $\langle u \mid \alpha v+\beta w\rangle=\alpha\langle u \mid v\rangle+\beta\langle u \mid w\rangle$
        \item[b.] $\langle\alpha u+\beta v \mid w\rangle=\alpha^{*}\langle u \mid w\rangle+\beta^{*}\langle v \mid w\rangle$
        \item[c.] $\langle u \mid v\rangle=\langle v \mid u\rangle^{*}$
    \end{enumerate}

\item[] \textbf{In-Class Exercise 4:} Review \href{https://www2.seas.gwu.edu/~simhaweb/quantum/modules/module2/problems2.html#vectors}{these examples} and then solve the following. Given $|u\rangle=\left(\frac{1}{\sqrt{2}}, \frac{i}{\sqrt{2}}\right)$, $|v\rangle=\left(\frac{1}{\sqrt{2}}, -\frac{i}{\sqrt{2}}\right)$, $w=(1,0)$, $\alpha=i$, $\beta=-\sqrt{2}i$, calculate (without converting $\sqrt{2}$ to decimal format):

    \begin{enumerate}
        \item[a.] $\alpha|v\rangle$, $|\alpha v\rangle$, $\langle\alpha u|$, $\alpha^{*}\langle u$;
        \item[b.] $\langle u \mid v\rangle$;
        \item[c.] $|u\rangle\langle v|,| u\rangle\langle v|| w\rangle$, and $\langle v \mid w\rangle|u\rangle$, and compare the latter two results;
        \item[d.] $\| u\rangle\left.\right|^{2},\langle u \mid u\rangle$;
        \item[e.] $\langle w|\alpha| u\rangle+\beta|v\rangle\rangle \text { and } \alpha\langle w \mid u\rangle+\beta\langle w \mid v\rangle$;
        \item[f.] $\langle\alpha \mid u\rangle+\beta|v\rangle|w\rangle \text { and } \alpha *\langle u \mid w\rangle+\beta^{*}\langle v \mid w\rangle$.
    \end{enumerate}

\item[] \textbf{In-Class Exercise 5:} Suppose $|u\rangle=(1,0,0,0)$, $|v\rangle=(0,0,1,0)$ and $W=\operatorname{span}(|u\rangle,|v\rangle)$.

    \begin{enumerate}
        \item[a.] Show that $|u\rangle$, $|v\rangle$ are linearly independent.
        \item[b.] What is the dimension of $W$?
        \item[c.] Express $|x\rangle=(-i, 0, i+1,0)$ in terms of $|u\rangle$, $|v\rangle)$.
        \item[d.] Is $|u\rangle$, $|v\rangle)$ a basis for $W$? Explain.
        \item[e.] Is $W=\mathbb{C}^{k}$ for any value of $k$?
    \end{enumerate}

\item[] \textbf{In-Class Exercise 6:} For the vectors $|00\rangle$, $|01\rangle$, $|10\rangle$, $|11\rangle$ defined above, compute the outer products

    \begin{enumerate}
        \item[a.] $|00\rangle\langle 00|$
        \item[b.] $|01\rangle\langle 01|$
        \item[c.] $|10\rangle\langle 10|$
        \item[d.] $|11\rangle\langle 11|$
    \end{enumerate}

\item[] \textbf{In-Class Exercise 7:} Using the vectors $|0\rangle$, $|1\rangle$, compute the vectors

    \begin{enumerate}
        \item[a.] $\left|h_{1}\right\rangle=\frac{1}{\sqrt{2}}(|0\rangle+|1\rangle)$
        \item[b.] $\left|h_{2}\right\rangle=\frac{1}{\sqrt{2}}(|0\rangle-|1\rangle)$
        \item[c.] Show that $\left|h_{1}\right\rangle$, $\left|h_{2}\right\rangle$ are orthonormal (unit length and orthogonal).
        \item[d.] Show that $\left|h_{1}\right\rangle$, $\left|h_{2}\right\rangle$ are a basis for $\mathbb{C}^2$. [Hint: start by expressing $|0\rangle$, $|1\rangle$ in terms of $\left|h_{1}\right\rangle,\left|h_{2}\right\rangle$.]   
        \end{enumerate}

\item[] \textbf{In-Class Exercise 8:} For any projector $|v\rangle\langle v|$ show that $(|v\rangle\langle v|)^{\dagger}=|v\rangle\langle v|$. [Hint: you can transpose and then conjugate.] Note: a matrix $A$ like $P_{v}=|v\rangle\langle v|$ that satisfies $A^{\dagger}=A$ is called \emph{Hermitian}, as we'll see below.

\item[] \textbf{In-Class Exercise 9:} Using the vectors $|0\rangle$, $|1\rangle$, compute the vectors

    \begin{enumerate}
        \item[a.] $\left|y_{1}\right\rangle=\frac{1}{\sqrt{2}}|0\rangle+\frac{i}{\sqrt{2}}|1\rangle$
        \item[b.] $\left.\left|y_{2}\right\rangle=\frac{1}{\sqrt{2}}|0\rangle-\frac{i}{\sqrt{2}}|1\rangle\right)$
        \item[c.] Show that $\left|y_{1}\right\rangle$, $\left|y_{2}\right\rangle$ are a basis for $\mathbb{C}^{2}$.
        \item[d.] Show the completeness relation for this basis.
    \end{enumerate}

\item[] \textbf{In-Class Exercise 10:} Consider these matrices. (Yes, the latter two have special names.)

    $$
    A=\left[\begin{array}{cc}
    1 & -i \\
    i & 1
    \end{array}\right] \quad B=\left[\begin{array}{cc}
    i & 1 \\
    1 & -i
    \end{array}\right] \quad Y=\left[\begin{array}{cc}
    0 & -i \\
    i & 0
    \end{array}\right] \quad H=\left[\begin{array}{cc}
    \frac{1}{\sqrt{2}} & \frac{1}{\sqrt{2}} \\
    \frac{1}{\sqrt{2}} & -\frac{1}{\sqrt{2}}
    \end{array}\right]
    $$
    
    Compute
    
    \begin{enumerate}
        \item[1.] The adjoints $A^{\dagger}$, $B^{\dagger}$, $Y^{\dagger}$, $H^{\dagger}$
        \item[2.] $A^{\dagger} A, Y^{\dagger} Y \text{ and } H^{\dagger} H$
        \item[3.] $A A^{\dagger}, Y Y^{\dagger}$ and $H H^{\dagger}$
    \end{enumerate}
    
    In computing the adjoint, do we get the same result if we first apply conjugation and then transpose?

\item[] \textbf{In-Class Exercise 11:} Use

    $$A=\left[\begin{array}{cc}1 & -i \\ i & 1\end{array}\right] \quad B=\left[\begin{array}{cc}i & 1 \\ 1 & -i\end{array}\right] \quad \alpha=1+i$$
    
    as examples in confirming (i)-(iv) above.

\item[] \textbf{In-Class Exercise 12:} If $\alpha$ were not real above, show with an $2 \times 2$ example that $\alpha A$ is not Hermitian when $A$ is Hermitian.

\item[] \textbf{In-Class Exercise 13:} Use a $2 \times 2$ example to show that the sum of two unitary matrices is not necessarily unitary.

\item[] \textbf{In-Class Exercise 14:} Suppose $A=\left[\begin{array}{ll}2 & 0 \\ 0 & 3\end{array}\right]$ and $|u\rangle=\alpha\left|e_{1}\right\rangle+\beta\left|e_{2}\right\rangle$. Show that $\langle u|A| u\rangle=2|\alpha|^{2}+3|\beta|^{2}$.

\item[] \textbf{In-Class Exercise 15:} Using $\left|v_{1}\right\rangle,\left|v_{2}\right\rangle$ as the basis, 

    \begin{enumerate}
        \item[a.] Find the coordinates of $\left|e_{1}\right\rangle,\left|e_{2}\right\rangle$;
        \item[b.] Then, compute $\left|y_{2}\right\rangle=P_{e_{2}}|u\rangle$ in this basis.
        \item[c.] Check your calculations by obtaining $\left|y_{2}\right\rangle$ in the standard basis and converting that to the $\left|v_{1}\right\rangle,\left|v_{2}\right\rangle$ basis.
    \end{enumerate}

\item[] \textbf{In-Class Exercise 16:} This is just a writing exercise to get used to the new symbols. Rewrite the previous exercise with the new symbols. That is, using $|+\rangle,|-\rangle$ as the basis, 

    \begin{enumerate}
        \item[a.] Find the coordinates of $|0\rangle,|1\rangle$;
        \item[b.] Then, compute $\left|y_{2}\right\rangle=P_{0}|u\rangle$ in this basis.
        \item[c.] Check your calculations by obtaining $\left|y_{2}\right\rangle$ in the standard basis and converting that to the $|+\rangle,|-\rangle$ basis.
    \end{enumerate}

\item[] \textbf{In-Class Exercise 17:} Suppose A is an operator that "flips" the Hadamard basis vectors. That is, $A|+\rangle=|-\rangle$ and $A|-\rangle=|+\rangle$. Use the approach above to derive the matrix for $A$. Then, apply the matrix to $|\psi\rangle=\alpha|0\rangle+\beta|1\rangle$. Is the resulting vector orthogonal to $|\psi\rangle$?

\item[] \textbf{In-Class Exercise 18:} Show that $|z\rangle \in W$.

\item[] \textbf{In-Class Exercise 19:} What is the distribution (histogram) expected if both slits are open and if light is made of particles? What is the height of the distribution at the center?

\item[] \textbf{In-Class Exercise 20:} For a given photon, what is the probability that detector RT goes off? What are the probabilities for the other detectors?

\item[] \textbf{In-Class Exercise 21:} For a given photon, what are the probabilities for detecting at D1 vs D2? Does the adjustable distance matter?

\item[] \textbf{In-Class Exercise 22:} Based only on the reasoning in the previous two exercises, what is the probability of detection at D1? At D2?

\end{enumerate}
\end{document}