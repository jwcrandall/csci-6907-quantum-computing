\documentclass[main.tex]{subfiles}
\begin{document}

\href{https://www2.seas.gwu.edu/~simhaweb/quantum/modules/module3/module3.html}{Module 3: The Single Qubit}

\begin{enumerate}

\item[] \textbf{In-Class Exercise 1:} Suppose the qubit state is $|\psi\rangle=|+\rangle$ and the measurement basis is: $|w\rangle=\frac{\sqrt{3}}{2}|0\rangle-\frac{1}{2}|1\rangle$, and $\left|w^{\perp}\right\rangle=\frac{1}{2}|0\rangle+\frac{\sqrt{3}}{2}|1\rangle$. \textbf{Q.} What are the possible output vectors, and with what probabilities do they occur? Start by expressing $|+\rangle$ in the $|w\rangle$, $\left|w^{\perp}\right\rangle$ basis by computing the coefficients $\langle w \mid+\rangle,\left\langle w^{\perp} \mid+\right\rangle$. \textbf{A.} 
\begin{align*}
    |\psi\rangle                    & = \langle w \mid+\rangle |0\rangle +  \langle w^{\perp} \mid+\rangle | 1 \rangle\\
    \langle w \mid+\rangle          & = \frac{\sqrt{3}}{2}\langle 0 |- \frac{1}{2} \langle 1| \mid | + \rangle\\
                                    & = \left[ \begin{array}{ll} \frac{\sqrt{3}}{2} & -\frac{1}{2} \end{array} \right]
                                    \left[ \begin{array}{l} \frac{1}{\sqrt{2}} & \frac{1}{\sqrt{2}} \end{array} \right]\\
                                    & = \frac{\sqrt{3}-1}{2\sqrt{2}}\\
    \langle w^{\perp} \mid+\rangle  & = \frac{1}{2}\langle 0 | + \frac{\sqrt{3}}{2} \langle 1| \mid | + \rangle\\
                                    & = \left[ \begin{array}{ll} \frac{1}{2} & + \frac{\sqrt{3}}{2} \end{array} \right]
                                    \left[ \begin{array}{l} \frac{1}{\sqrt{2}} & \frac{1}{\sqrt{2}} \end{array} \right]\\
                                    & = \frac{\sqrt{3}+1}{2\sqrt{2}}\\
    |0\rangle                       & = \left| \frac{\sqrt{3}-1}{2\sqrt{2}} \right|^2 \tag{output vector}\\ 
                                    & =\frac{2-\sqrt{3}}{4} \tag{probability outcome}\\
    |1\rangle                       & = \left| \frac{\sqrt{3}+1}{2\sqrt{2}} \right|^2 \tag{output vector}\\
                                    & =\frac{2+\sqrt{3}}{4} \tag{probability outcome}
\end{align*}

\item[] \textbf{In-Class Exercise 2:} \textbf{Q.} What does the $X$ gate do to the state $|\psi\rangle=\alpha|0\rangle+\beta|1\rangle$? \textbf{A.} 
\begin{align*}
    X |\psi\rangle  & = \alpha X |0\rangle + \beta X|1\rangle\\
                    & =  \alpha \left[\begin{array}{ll} 0 & 1 \\ 1 & 0 \end{array}\right] \left[\begin{array}{l} 1 \\ 0 \end{array} \right]
                    + \beta \left[\begin{array}{ll} 0 & 1 \\ 1 & 0 \end{array}\right] \left[\begin{array}{l} 0 \\ 1 \end{array} \right]\\
                    & = \beta |0\rangle + \alpha |1\rangle\\
\end{align*}

\item[] \textbf{In-Class Exercise 3:} \textbf{Q.} Derive the above result that applies $H$ to $|\psi\rangle=\alpha|0\rangle+\beta|1\rangle$. Also show that $H|\psi\rangle=\alpha|+\rangle+\beta|-\rangle$. \textbf{A.}
\begin{align*}
    H | \psi \rangle    & = H \left( \alpha | 0 \rangle + \beta | 1 \rangle \right) \\
                        & = \left[\begin{array}{ll} \frac{1}{\sqrt{2}} & \frac{1}{\sqrt{2}} \\ \frac{1}{\sqrt{2}} & -\frac{1}{\sqrt{2}} \end{array} \right] 
                        \left[\begin{array}{l} \alpha \\ \beta \end{array} \right]\\
                        & =  \left[\begin{array}{l} \frac{\alpha + \beta}{\sqrt{2}} \\ \frac{\alpha - \beta}{\sqrt{2}} \end{array} \right] \\
                        & = \frac{\alpha+\beta}{\sqrt{2}}|0\rangle+\frac{\alpha-\beta}{\sqrt{2}}|1\rangle \\
    H | \psi \rangle    & = \alpha H | 0 \rangle + \beta H | 1 \rangle \\
                        & =  \alpha \left[\begin{array}{ll} \frac{1}{\sqrt{2}} & \frac{1}{\sqrt{2}} \\ \frac{1}{\sqrt{2}} & -\frac{1}{\sqrt{2}} \end{array}\right] \left[\begin{array}{l} 1 \\ 0 \end{array} \right]
                        + \beta \left[\begin{array}{ll} \frac{1}{\sqrt{2}} & \frac{1}{\sqrt{2}} \\ \frac{1}{\sqrt{2}} & -\frac{1}{\sqrt{2}} \end{array}\right] \left[\begin{array}{l} 0 \\ 1 \end{array} \right]\\
                        & =  \alpha \left[\begin{array}{l} \frac{1}{\sqrt{2}} \\ \frac{1}{\sqrt{2}} \end{array} \right]
                        + \beta \left[\begin{array}{l} \frac{1}{\sqrt{2}} \\ -\frac{1}{\sqrt{2}} \end{array} \right]\\
                        & = \alpha | + \rangle + \beta | - \rangle
\end{align*}

\item[] \textbf{In-Class Exercise 4:} Consider the two circuits below, each given the same input.

    \begin{enumerate}
        \item[1.] Write down the possible states of the outputs.
        \item[2.] Calculate the probabilities associated with each output state.
    \end{enumerate}

\item[] \textbf{In-Class Exercise 5:} Consider the following set up: (polariser 6)

What percentage of $|1\rangle$ photons arriving on the left reach the output? Work through your probability calculations as shown above.

\item[] \textbf{In-Class Exercise 6:}

    \begin{enumerate}
        \item[1.] What would go wrong if the S-H strings were exchanged before Bob performs qubit measurements?
        \item[2.] Write out the details of Case 2(B) above, explaining the details of measurement and probabilities.
    \end{enumerate}

\item[] \textbf{In-Class Exercise 7:} Show that with $a,|\psi\rangle$ and $\left|\psi^{\prime}\right\rangle$ defined above,

    \begin{enumerate}
        \item[1.] $|a|^{2}=1$
        \item[2.] $\left|\psi^{\prime}\right\rangle=a|\psi\rangle$
        \end{enumerate}

\item[] \textbf{In-Class Exercise 8:} Write down the two projector matrices $P_{v}, P_{v^{\perp}}$ for the general 2D basis $|v\rangle=a|0\rangle+b|1\rangle$, and $\left|v^{\perp}\right\rangle=b^{*}|0\rangle-a^{*}|1\rangle$ and check your calculations by showing $P_{v} P_{v^{\perp}}=0$.

\item[] \textbf{In-Class Exercise 9:} Suppose the qubit state is $|\psi\rangle=|+\rangle$ and the measurement basis is: $|w\rangle=\frac{\sqrt{3}}{2}|0\rangle-\frac{1}{2}|1\rangle$, and $\left|w^{\perp}\right\rangle=\frac{1}{2}|0\rangle+\frac{\sqrt{3}}{2}|1\rangle$. Apply the projective measurement approach:

    \begin{enumerate}
        \item[1.] Compute the projectors $P_{w}, P_{w^{\perp}}$.
        \item[2.] Apply the three steps, simplifying where possible.
    \end{enumerate}
    
    Confirm that you get the same results as in Exercise 1.

\item[] \textbf{In-Class Exercise 10:} Use the projector matrices for the basis and the numbers $\gamma_{1}=3, \gamma_{2}=5$ to construct the Hermitian. Then show that these are the eigenvalues with $k t+,|-\rangle$ as eigenvectors.

\item[] \textbf{In-Class Exercise 11:} Show that the Hadamard matrix can be written as $H=\frac{1}{\sqrt{2}}(|0\rangle\langle 0|+| 0\rangle\langle 1|+| 1\rangle\langle 0|-| 1\rangle\langle 1|)$ and then apply it to $|\psi\rangle=\alpha|0\rangle+\beta|1\rangle$.

\end{enumerate}
\end{document}