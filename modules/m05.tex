\documentclass[main.tex]{subfiles}
\begin{document}

\href{https://www2.seas.gwu.edu/~simhaweb/quantum/modules/module5/module5.html}{Module 5: Multiple Qubits - Measurement and Modification}\\
\href{https://www2.seas.gwu.edu/~simhaweb/quantum/modules/module5/problems5.html}{Module-5 Solved Problems}

\begin{enumerate}

\item[] \textbf{In-Class Exercise 1:} Consider the single-qubit projectors $P_{+}=|+\rangle\langle+|$ and $P_{-}=|-\rangle\langle-|$.
    \begin{enumerate}
        \item[1.] Show that $\left(P_{+} \otimes I\right)|00\rangle=\frac{1}{\sqrt{2}}|+, 0\rangle$ and $\left(P_{+} \otimes I\right)|11\rangle=\frac{1}{\sqrt{2}}|+, 1\rangle$
        \item[2.] Show that $\frac{1}{\sqrt{2}}|+, 0\rangle+\frac{1}{\sqrt{2}}|+, 1\rangle=|+,+\rangle$
        \item[3.] Expand $\left(P_{+} \otimes I\right)$ into a matrix and use that to show $\left(P_{+} \otimes I\right)|\psi\rangle=|+,+\rangle$ where $|\psi\rangle=\frac{1}{\sqrt{2}}(|00\rangle+|11\rangle)$
    \end{enumerate}

\item[] \textbf{In-Class Exercise 2:} Apply the projectors $\left(P_{+} \otimes I\right)$ and $\left(P_{-} \otimes I\right)$ to $|\psi\rangle=\frac{1}{2}(|00\rangle+|01\rangle+|10\rangle+|11\rangle$ using the Dirac-notation approach above, and then confirm your results by expanding into matrices.

\item[] \textbf{In-Class Exercise 3:} Fill in the missing step of normalization in the above example. Then, apply the same projectors to the vector $|\psi\rangle=|+,+,+\rangle$ and list the outcomes and probabilities.

\item[] \textbf{In-Class Exercise 4:} There was one step missing in the sequence of projections above: what happened to the terms $\sqrt{|\alpha|^{2}+|\beta|^{2}}$ and $\sqrt{|\gamma|^{2}+|\delta|^{2}}$ ? Hint: write down the probabilities  $\left.\left.\left|P_{0}\right| \psi\right\rangle\left.\right|^{2}$ and $\left|P_{2}\right| \psi_{0}\right\rangle\left.\right|^{2}$, and then ask: what product of probabilities leads us to observing $|00\rangle$ as the outcome?

\item[] \textbf{In-Class Exercise 5:}
\begin{enumerate}
    \item[1.] Show that $X^{2}=I$ both with matrix and Dirac forms.
    \item[2.] Write out the matrix and Dirac forms of $X \otimes X$.
\end{enumerate}

\item[] \textbf{In-Class Exercise 6:} Confirm the above result with the matrix form of $(I \otimes X)|\psi\rangle$.

\item[] \textbf{In-Class Exercise 7:} Write down the matrices for $H^{\otimes 2}$ and  $H^{\otimes 3}$ in the above form (pulling out $\frac{1}{\sqrt{2^{n}}}$). This is admittedly a bit tedious but will be useful when we analyze the structure of this matrix for algorithmic insight.

\item[] \textbf{In-Class Exercise 8:}
    \begin{enumerate}
        \item[1.] Use the Dirac form to apply $C_{\text{NOT}}$ to $|\psi\rangle=(\alpha|0\rangle+\beta|1\rangle) \otimes |0\rangle$ and $\left|\psi^{\prime}\right\rangle=(\alpha|0\rangle+\beta|1\rangle) \otimes |1\rangle$. 
        \item[2.] Then confirm with matrix versions.
        \item[3.] Apply this to the case where $\alpha=\beta=\frac{1}{\sqrt{2}}$.
    \end{enumerate}

\item[] \textbf{In-Class Exercise 9:} Use the above ideas to draw a two-stage circuit that takes the vector $|00\rangle$ to $\left|\Phi^{+}\right\rangle$.

\item[] \textbf{In-Class Exercise 10:} Show the steps in applying $C_{\text{NOT}}$ to each of: $|-,+\rangle,|+,-\rangle,|-,-\rangle$.


\item[] \textbf{In-Class Exercise 11:}
\begin{enumerate}
    \item [1.] Show that $(H \otimes I \otimes I) \frac{1}{\sqrt{2}}(\alpha|000\rangle+\alpha|011\rangle+\beta|110\rangle+\beta|101\rangle)=\frac{1}{2}(|00\rangle \otimes(\alpha|0\rangle+\beta|1\rangle)+|01\rangle \otimes(\beta|0\rangle+\alpha|1\rangle)+|10\rangle \otimes(\alpha|0\rangle-\beta|1\rangle)+|11\rangle \otimes(\alpha|1\rangle-\beta|0\rangle))$ Hint: see one of the solved problems
    \item [2.] Show that $(I \otimes I \otimes Z)(|10\rangle \otimes(\alpha|0\rangle-\beta|1\rangle))=|10\rangle(\alpha|0\rangle+\beta|1\rangle)$
\end{enumerate}

\end{enumerate}
\end{document}