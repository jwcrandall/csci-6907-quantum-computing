\documentclass[main.tex]{subfiles}
\begin{document}

\href{https://www2.seas.gwu.edu/~simhaweb/quantum/modules/module6/module6.html}{Module 6: Quantum Circuits - Gates}\\
\href{https://www2.seas.gwu.edu/~simhaweb/quantum/modules/module6/problems6.html}{Module-6 Solved Problems}

\begin{enumerate}

\item[] \textbf{In-Class Exercise 1:} Use the Dirac form and
\begin{enumerate}
    \item [1.] Show that $Y(\alpha|0\rangle+\beta|1\rangle)=i(\alpha|1\rangle-\beta|0\rangle)$
    \item [2.] Show that $Y$ switches the Hadamard basis vectors. (Remember global phase.)
\end{enumerate}

\item[] \textbf{In-Class Exercise 2:} Prove the identity $YZ=-ZY=iX$

\item[] \textbf{In-Class Exercise 3:} Derive the matrix for $R_{Z}(\theta)$ and show that it is unitary. Is it Hermitian?

\item[] \textbf{In-Class Exercise 4:} Show that the three operators $T, R, K$ commute with different parameters:
\begin{enumerate}
    \item [1.] $T\left(\alpha_{1}\right) T\left(\alpha_{2}\right)=T\left(\alpha_{2}\right) T\left(\alpha_{1}\right)$
    \item [2.] $R\left(\beta_{1}\right) R\left(\beta_{2}\right)=R\left(\beta_{2}\right) R\left(\beta_{1}\right)$
    \item [3.] $K\left(\delta_{1}\right) K\left(\delta_{2}\right)=K\left(\delta_{2}\right) K\left(\delta_{1}\right)$ 
\end{enumerate}

\item[] \textbf{In-Class Exercise 5:} Show that $P\theta$ is unitary. 

\item[] \textbf{In-Class Exercise 6:} Show that the same property that we derived for $Z$ can be derived for $X$, that is, $e^{-i \frac{\pi}{2} X}=-i X$. This means fractional powers of $X$ can be computed in the same way, as shown in one of the solved examples.

\item[] \textbf{In-Class Exercise 7:} Show the two $Y$ properties:
\begin{enumerate}
    \item [1.] $H Y H=-Y$
    \item [2.] $H R_{Y}(\theta) H=R_{Y}(-\theta)$
\end{enumerate}

\item[] \textbf{In-Class Exercise 8:} 
\begin{enumerate}
    \item [1.]  Compute $R_{Z}\left(\frac{\pi}{2}\right) R_{X}\left(\frac{\pi}{2}\right) R_{Z}\left(\frac{\pi}{2}\right)$
    \item [2.] Then show that $H=K\left(\frac{\pi}{2}\right) R_{Z}\left(\frac{\pi}{2}\right) R_{X}\left(\frac{\pi}{2}\right) R_{Z}\left(\frac{\pi}{2}\right)$
\end{enumerate}

\item[] \textbf{In-Class Exercise 9:} 
\begin{enumerate}
    \item [1.] Verify that $Y^{\frac{1}{2}} Z K\left(-\frac{\pi}{4}\right)=H$.
    \item [2.] Compute the three vectors $\left|\psi_{1}\right\rangle,\left|\psi_{2}\right\rangle,\left|\psi_{3}\right\rangle$ at each step.
\end{enumerate}

\item[] \textbf{In-Class Exercise 10:} Write down the Dirac and matrix forms of the two versions of the "upside-down", once with $|1\rangle$ achieving the flip, and once with $|0\rangle$. Show using Dirac notation that each is its own inverse. 

\item[] \textbf{In-Class Exercise 11:} Use matrices or Dirac forms to show both results above:
\begin{enumerate}
    \item [1.] $\bar{C}_{\mathrm{Z}}=C_{\mathrm{Z}}$
    \item [2.] $C_{\text {NOT }}=(I \otimes H) C_{Z}(I \otimes H)$
\end{enumerate}

\item[] \textbf{In-Class Exercise 12:}
\begin{enumerate}
    \item [1.] Use either matrices or Dirac forms to show SWAP $=C_{N O T} \bar{C}_{NOT} C_{NOT}$
    \item [2.] Sketch the staged circuit for swapping the first and eighth qubits. (Your drawing can be rough, and scanned from paper.)
    \item [3.] With $n=2^{m}$ qubits, how many SWAP gates are needed to swap the first and last qubits?
\end{enumerate}
 
\item[] \textbf{In-Class Exercise 13:} Complete the remaining two steps in showing $\left(I \otimes \bar{C}_{\text {NOT }}\right) C C_{\text {NOT }}\left(I \otimes \bar{C}_{\text {NOT }}\right)$.

\item[] \textbf{In-Class Exercise 14:} Use the above idea to draw the circuit that converts $|110\rangle$ to $|001\rangle$.

\end{enumerate}
\end{document}