\documentclass[main.tex]{subfiles}
\begin{document}

\href{https://www2.seas.gwu.edu/~simhaweb/quantum/modules/module10/module10.html}{Module 10: Algorithms, part I - oracle problems}

\begin{enumerate}

\item[] \textbf{In-Class Exercise 1:} In-Class Exercise 1: Suppose $f(x)=f\left(x_{1}, x_{2}, x_{3}\right)$ is a 3-input Boolean function that falls into one of two categories:
\begin{enumerate}
    \item [1.] $f$ is constant, meaning either $f(x)=0$ or $f(x)=1$ for all $x$.
    \item [2.] $f$ is balanced, meaning $f(x)=0$ for exactly half the inputs (4 input combinations in this case), but we don't know which ones. 
\end{enumerate}
How many trials are needed to determine which category an unknown $f$ falls in?

\item[] \textbf{In-Class Exercise 2:} In-Class Exercise 2: Show that this is true. That is, $U_{f_{2}}|+\rangle|-\rangle=|+\rangle|-\rangle$.

\item[] \textbf{In-Class Exercise 3:} In-Class Exercise 3: Show that this is true. That is, $U_{f_{4}}|+\rangle|-\rangle=|-\rangle|-\rangle$.

\end{enumerate}
\end{document}